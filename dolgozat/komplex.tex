\documentclass[a4paper]{paper}

% Set margins
\usepackage[hmargin=2cm, vmargin=2cm]{geometry}

\frenchspacing

% Language packages
\usepackage[utf8]{inputenc}
\usepackage[T1]{fontenc}
\usepackage[magyar]{babel}

% AMS
\usepackage{amssymb,amsmath}

% Graphic packages
\usepackage{graphicx}

% Colors
\usepackage{color}
\usepackage[usenames,dvipsnames]{xcolor}

% Enumeration
\usepackage{enumitem}

\begin{document}

\begin{center}
   \large \textbf{Hangvezérelt aszisztens asztali környezetekhez}
\end{center}

\vskip 1cm

\section{Feladat}

Beszédfelismerés módszertanánk megismerése. A rejtett Markov modell és a dinamikus idővetemítés alkalmazásával hangminták elemzése. Hangjellemzők kinyerése. Gépi tanulási módszerek alkalmazásával modell betanítása minták alapján. Asztali környezeti műveletek végrehajtásának megvalósítása egyszerű vezényszavak segítségével. Elkészült alkalmazás használhatóságának tesztelése, eredmények értékelése.

\section{Hangfeldolgozási módszerek}

% TODO: Leírni, hogy milyen célból, és nagyvonalakban hogy szokták feldolgozni.

\section{Dinamikus idővetemítés}

\section{Rejtett Markov Modell}

\section{Elérhető szoftveres eszközök áttekintése}

Praat
% https://en.wikipedia.org/wiki/Praat

Sphinx
% https://pypi.org/project/pocketsphinx/

\section{Szavak szegmentálása és felismerése}

\end{document}
